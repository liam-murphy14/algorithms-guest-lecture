\documentclass[11pt]{exam}
\usepackage[T1]{fontenc}
\usepackage[amsthm]{newpx}
\usepackage[margin=1in]{geometry}
\usepackage{mathtools}
\usepackage{float}
\usepackage{enumerate}
\usepackage{listings}
\usepackage{hyperref}
\usepackage[boxed]{algorithm}
\usepackage[noend]{algpseudocode}
\hypersetup{
    colorlinks=true,
    linkcolor=blue,
    filecolor=magenta,      
    urlcolor=cyan,
}

% in order to compile this file you need to get 'header.tex' from
% Canvas and change the line below to the appropriate file path
\input{./header}


\begin{document}
\section*{Knight's Path}

{\large Description}\\
Given an $n\times n$ chess board, a starting position, and ending position,
and a bishop position, compute the shortest path a knight starting at the
start position and ending at the end position can take without being
threatened by the bishop. If the knight can capture the bishop, then
it is free to take any path.\\
\textbf{Formally:} given an integer pair $(x_0,y_0)$ starting position, an integer pair $(x_1, y_1)$ ending position, and an integer pair $(p, q)$ bishop position, compute the shortest path from the start to the end without being threatened by the bishop, returning $-1$ if no path exists.\\\\
{\large Notes}\\
A knight can move in an L-shape. That is, given some position $(p,q)$ on
the board, the knight can move to $(p+1,q+2), (p+2,q+1), (p-1,q+2)$, etc.
Here is a picture for better visualization.
\begin{figure}[H]
	\centering
	\includegraphics[width=0.7\linewidth]{knight_moves.png}
	\caption{A knight's possible moves}
\end{figure}
The shortest path is one with the fewest of such moves.\\\
Now, the bishop moves on diagonals. That is, given a position $(p,q)$ on the board, the bishop threatens any position diagonal from $(p, q)$. Here is a picture for better visualization.
\begin{figure}[H]
	\centering
	\includegraphics[width=0.7\linewidth]{bishop_moves.png}
    \caption{A bishop's threatened positions}
\end{figure}


\newpage
\section*{Solution}
\textbf{Idea:} Treat the chessboard as a graph, and use a known graph traversal algorithm (BFS) to solve!
\begin{enumerate}
	\item Set up functions to construct nodes, check their validity, and create a list of nodes that the bishop threatens
	\item Create a BFS function that accounts for the possible threat of the bishop
	\item Wire up our constructions to solve the problem
	\item Optimize if time remains
\end{enumerate}
\section*{Set Up}
\begin{algorithm}[H]
\begin{algorithmic}
\Procedure{Make-Node}{$x,y,d$}
	\Return{$(x,y,d)$}
\algorithmiccomment{{\it A Node is a 3-Tuple with an x coordinate, y coordinate, and integer distance}}
\EndProcedure
\Procedure{Is-Valid}{$x,y,n$}
	\If{$x<1$ or $y<1$ or $x>n$ or $y>n$}
		\Return{false}
	\EndIf
	\Return{true}
\EndProcedure
\Procedure{Is-Valid-Bishop}{$x,y,A$}
	\If{$(x,y)$ in $A$}
		\Return{false}
	\EndIf
	\Return{true}
\EndProcedure
\Procedure{Bishop-Positions}{$x,y,n$}
	\State{$A\gets\varnothing$}
	\For{$i\in\set{1,2,\ldots,n}$}
		\If{\Call{Is-Valid}{$x+i,y+i,n$}}
			\State{$A.add(x,y)$}
		\EndIf
		\If{\Call{Is-Valid}{$x-i,y+i,n$}}
			\State{$A.add(x,y)$}
		\EndIf
		\If{\Call{Is-Valid}{$x+i,y-i,n$}}
			\State{$A.add(x,y)$}
		\EndIf
		\If{\Call{Is-Valid}{$x-i,y-i,n$}}
			\State{$A.add(x,y)$}
		\EndIf
	\EndFor
	\Return{$A$}
\EndProcedure
\end{algorithmic}
\end{algorithm}
% We want to design an algorithm that treats the chessboard as a graph, then use BFS to find the shortest path (with a few modifications, of course).
% However, we can't just make the entire chessboard, a graph, because the
% knight can't move to just any space! Here's the idea: use BFS to determine
% how many moves it would take to reach the bishop, capture, then move to the
% ending position (or if this is even possible) and compare it with just
% avoiding the bishop's threat and going straight to the end position (if
% this is even possible). Return the length of the shorter path, if either
% answer reaches the end space.\\\\
% {\large 1. Set up}\\
% First, we need to set up the framework that will allow us to treat the chessboard as a graph. We will define three functions. The first is a ``constructor'' of sorts to create nodes, which are 3-Tuples consisting of row and column coordinates, and a distance, which will represent the distance (in knight moves) from the first space. The second checks if a node is valid (i.e. on our chessboard). The third creates a set of nodes threatened by the bishop. Recall that if the bishop is on a square $(x,y)$, then it threatens all squares $\set{(x\pm i, y\pm i):i=1,2,\ldots}$.
\section*{Modified BFS}
\begin{algorithm}[H]
\begin{algorithmic}
\Procedure{Shortest-Path}{$x_0,y_0,x_1,y_1,n,b,A$}
\algorithmiccomment{{\it $x_0,y_0$ is start position, $x_1,y_1$ is end position, $n$ is dimension, and $b$ is a Boolean for if the bishop exists}}
	\State{$row\gets\set{2,2,-2,-2,1,1,-1,-1}$}
	\State{$col\gets\set{-1,1,1,-1,2,-2,2,-2}$}
	\State{$visited\gets\varnothing$}
	\State{Queue $Q\gets \{\Call{Make-Node}{x_0,y_0,0}\}$}
	\While{$Q\neq\varnothing$}
		\State{$v\gets Q.pop$}
		\If{$v.x=x_1$ and $v.y=y_1$}
			\Return{$v.d$}
		\EndIf
		\If{$v\notin visited$}
			\State{$visited.add(v)$}
			\For{$i\in\set{1,2,\ldots,8}$}
				\State{$x_{new}\gets v.x+col[i]$}
				\State{$y_{new}\gets v.y+row[i]$}
				\If{$b=true$}
					\If{\Call{Is-Valid}{$x_{new},y_{new},n$} and \Call{Is-Valid-Bishop}{$x_{new},y_{new},A$}}
						\State{$Q.add(\Call{Make-Node}{x_{new},y_{new},v.d+1})$}
					\EndIf
				\Else
					\If{\Call{Is-Valid}{$x_{new},y_{new},n$}}
						\State{$Q.add(\Call{Make-Node}{x_{new},y_{new},v.d+1})$}
					\EndIf
				\EndIf
			\EndFor
		\EndIf
	\EndWhile
	\Return{-1}
\EndProcedure
\end{algorithmic}
\end{algorithm}
\section*{Final}
\begin{algorithm}[H]
\begin{algorithmic}
\Procedure{Knight's-Path}{$x_0,y_0,x_1,y_1,b_x,b_y,n$}
	\State{$A\gets$ \Call{Bishop-Positions}{$b_x,b_y,n$}}
	\State{to-bishop $\gets$ \Call{Shortest-Path}{$x_0,y_0,b_x,b_y,n,true,A$}}
	\State{to-goal $\gets$ \Call{Shortest-Path}{$x_0,y_0,x_1,y_1,n,true,A$}}
	\If{to-bishop $\neq -1$}
		\State{b-to-goal $\gets$ \Call{Shortest-Path}{$b_x,b_y,x_1,y_1,n,false,A$}}
		\If{b-to-goal $\neq -1$}
			\If{to-goal $\neq -1$}
				\Return{$\min(\text{to-bishop}+\text{b-to-goal},\text{to-goal})$}
			\Else\ 
				\Return{$\text{to-bishop}+\text{b-to-goal}$}
			\EndIf
		\Else\  
			\Return{to-goal}
		\EndIf
	\Else\  
		\Return{to-goal}
	\EndIf
\EndProcedure
\end{algorithmic}
\end{algorithm}
\section*{Optimizations}
While this solution should be able to handle pretty large instances of the problem, what if we get a really, really big one? If the start and end are sufficiently far away, our algorithm may take too long to find a solution (at least, too long for hackerrank). How can we optimize?\\
\textbf{Idea:} Run a modified BFS from either end, and see if/where they meet!
\end{document}
